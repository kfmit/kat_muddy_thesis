%% ABSTRACT %%
\begin{center}
{\large \@title} \\
\emph{\footnotesize by} \\
\@author \\
\end{center}

\vspace{-1em}

\begin{center}
\begin{singlespace}
{\parindent0pt
\small
Submitted to the Department of Mechanical Engineering and the Joint Program in Applied Ocean Science and Engineering on \@date ~in partial fulfillment of the requirements for the degree of Master of Science in Mechanical Engineering}
\end{singlespace}
\end{center}

\begin{singlespace}
{\parindent0pt 
{\large \textsc{Abstract}} \\ %less than 200 words for WHOI (350 for MIT) 
	% 154 words
This thesis encompasses an analysis of underwater ambient noise collected by the yearlong Canada Basin Acoustic Propagation Experiment (CANAPE) on the Chukchi Shelf of the Arctic. This location contained the Beaufort Duct, a significant effect of climate change on the Arctic's underwater soundscape. A study of the statistical and probability metrics was conducted on a frequency band of 50-1900 Hz to examine the relation between environmental drivers and noise patterns. The presence of ice typically decreases broadband ambient noise, when compared to ice-free seas. However, the Beaufort Duct under ice increases the ambient noise levels below 1 kHz. The relationship between ambient noise and the environment is further explored by studying the link between distant ice movements and ambient levels Correlation between the two is found to exist from 300-1500 Hz, as distant (~500 km) ice drift motion appears to drive noise levels at the receiver. [Work supported by the Office of Naval Research]


\noindent Thesis Supervisor: Dr. Julien Bonnel \\
\noindent Title: Associate Scientist with Tenure, WHOI
}
\end{singlespace}

\newpage
\null
%\thispagestyle{empty}
\newpage