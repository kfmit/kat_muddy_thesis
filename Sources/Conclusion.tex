
%%%%%%%%%%%%%%%%%%%%%%%%%%%%%%%%%%%%%%%%%%%%%%%%%%%%%%%%%%%%%%%%%%%%%%%%
\chapter{Conclusions and Future Work}
%%%%%%%%%%%%%%%%%%%%%%%%%%%%%%%%%%%%%%%%%%%%%%%%%%%%%%%%%%%%%%%%%%%%%%%%

%%%%%%%%%%%%%%%%%%%%%%%%%%%%%%%%%%%%%%%%%%%%%%%%%%%%%%%%%%%%%%%%%%%%%%%%
\section{Conclusions from Statistic and Spacial Analysis}
%%%%%%%%%%%%%%%%%%%%%%%%%%%%%%%%%%%%%%%%%%%%%%%%%%%%%%%%%%%%%%%%%%%%%%%%

Exploring the probability and statistical metrics of a broad range of frequencies from 50-1900 Hz improves our understanding of the changing Arctic soundscape. If current warming trends continue, soon the rest of the Arctic may resemble the badly affected Canada Basin in terms of ice loss. Thinking of the Chukchi Shelf as representative of the Beaufort Sea and the Arctic at whole allows for the consideration of the results from this study being found beyond the experimental area.  In focusing on the effects and drivers of ambient noise, we gain some generalizations about the future of the ambient soundscape.

The majority of frequencies in this study were highly correlated with each other in ANL. This suggests that these frequencies have a shared acoustic driver or drivers. Looking at different frequencies in time also showed matching oscillations of ANL, though amplitudes differed by frequency. From looking at the statistics and probabilities of 50-1900 Hz ANL for 'ice with duct', 'ice without duct', and 'no ice', of number of conclusions are gathered. 

Over the course of a year, the ambient levels crossed a 45 dB range of noise shared among the three environments. Maximum noise levels fell as frequency increased, making 55 dB the quietest ANL(s) this area of the Arctic experiences when ice is present. The consistently loudest environment was 'no ice', reaching up to 90 dB at low frequency. Expanding the frequency band showed that 'no ice' shared only 30\% of its  ANL distribution >300 Hz with the 'ice' conditions. We gather that ANL(s) are more similar between lower frequencies.

Significantly, the Beaufort Duct has a pronounced effect on ANL when compared to a ductless environment. A 2-10 dB increase in ANL occurs when the Duct is in the area, but only for frequencies below 1000 Hz. Above 1000 Hz, the probability distributions of 'ice with duct' and 'ice without duct' become indistinguishable. This means the Duct has a greater effect on sounds and signals produced below 1000 Hz, improving their long range conduction underwater. Its continued existence would raise ambient levels across the Arctic as the Beaufort Duct increases its spacial and temporal range.

The sources of this ambient noise are non trivial and complex, creating sound capable of travelling many kilometers underwater. Though it is quieter under ice, there is still plenty of noise created by the ice itself as well as biologic factors. Ice drift movement magnitude and ANL share a strong relationship and correlation through time. This correlation is also shared from 300-1500 Hz, enforcing the idea of frequency correlated noise sharing sources. Correlation can be found across multiple hydrophones at different locations, validating this assumption.

These correlation spreads between IDM and ANL cover thousand of square kilometers with maximum distances beyond 1200 km.  While ANL(s) are known to decrease with frequency, the relationship between IDM and ANL does not fluctuate highly. Correlation $\geq0.4$ exists through all time and frequencies (300-1500 Hz), even as the size and shape of the map spread change. The movement of all correlation maps travels from west to east over time, reflecting the influence of environmental factors like the Alaskan Coastal Current and the Beaufort Gyre. 

IDM is not the only factor behind ambient noise, as other events like melting and calving affect ANL over the duration of the experiment. The conditions of the Arctic are highly favorable for long distance underwater sound through the Arctic facilitated by both the ducts in the water columns and the type of noise radiating from the ice. As these results come from hydrophones within the Beaufort Duct, the levels detected are probably local to the duct's depth range. However, as the Duct exists over much of the Arctic almost all year, increases in noise are not limited to one area in space. The noise levels in the Chukchi Shelf and Canada Basin may be representative of the changing noise levels in the Arctic at large. 
%though about being in duct makes effect bigger



%%%%%%%%%%%%%%%%%%%%%%%%%%%%%%%%%%%%%%%%%%%%%%%%%%%%%%%%%%%%%%%%%%%%%%%%

%%%%%%%%%%%%%%%%%%%%%%%%%%%%%%%%%%%%%%%%%%%%%%%%%%%%%%%%%%%%%%%%%%%%%%%%
\section{Impacts on the Arctic}

\subsection{Biological and Habitat Impact}

The changing acoustic soundscape of the Arctic is already known to have a profound effect on the biology that inhabits the polar seas. The increase in noise that the Beaufort Duct created could affect the critical ecosystem of the Arctic by changing the habits of the life within. Animals that use acoustic communications may change their behavioral patterns to accommodate the duct. Bowhead whales use a frequency band of up to 3000 Hz \parencite{clark1984sounds} and are known to dive around the Beaufort Duct's depth \parencite{simon2009behaviour}. The Beaufort Duct could amplify these whale's acoustic communication ranges, which could for the interaction of more whale pods. However, with higher levels of ambient noise, whale calls would need to be louder. Whether this signal-to-noise ratio compensation is attainable or beneficial to the bowhead population has not yet been studied.

The effects of airgun \parencite{halliday2020potential} and shipping noise \parencite{halliday2017potential} on marine mammals have been studied more than the behavioral response of other vertebrates like fish. Of the many species that reside in the Arctic and bolster the world's fishing economy, few have been studied. Exploring the relationship between shipping and Arctic sculpin was the first study done of this type. \parencite{ivanova2018sculpin} It's now known that shipping drives away Arctic cod \parencite{ivanova2020shipping}, but they are not the only species present. Migratory species like Arctic char that follow ice breakup \parencite{hammer2021char} may have their patterns disrupted by intrusive sound. Though studies have shown that noise pollution negatively affects invertebrates in other oceans \parencite{difranco2020}, little is known about the specific repercussions of noise on Arctic invertebrates.

The continued decrease in ice coverage draws those interested in the natural resources of the Arctic. Increases in noise generated from shipping, drilling, construction \parencite{gering2020aca} and more would exasperate the effects on a region already coping with increased ANL overall. Besides the increased noise, these attempts to exploit the Arctic would result in the destruction of habitats and potential release of substances harmful to the environment, like oil or chemical runoff.


\subsection{Underwater Communication}

There are some utilitarian aspects to the existence of the Beaufort Duct, if it continues to pervade the Arctic ocean over the years. The deeper sound duct opens the door to increased long range underwater communication at lower power than before \parencite{freitag2015underwatercomms} . As ambient noise is driven by cryogenic factors from far away (>500 km); the Beaufort Duct is already known to be conducive to long range proportion. This opens the door to various scientific and anthropogenic-related interests.

The Arctic will need continued monitoring as the effects of climate change continue to reduce the ice pack. This naturally leads to a waterfall of other changes in temperature, chemical makeup, and more which all affect how the Arctic can be studied. Environmental factors will need to be measured regularly in order to keep an active and correct picture of the Arctic, needed for any operations in this area \parencite{Schmidt2016commnav} . Proposals for monitoring arrays, float networks, and autonomous vehicles could benefit from the Duct's sound conducting properties by improving communications and reducing power consumption \parencite{kukulya2016development}. The increased human presence would certainly increase the amount of likely obtrusive man-made sound.



%%%%%%%%%%%%%%%%%%%%%%%%%%%%%%%%%%%%%%%%%%%%%%%%%%%%%%%%%%%%%%%%%%%%%%%%
\section{Potential for Future Work}


For this project, some transmission loss estimates were made using the modal propagation tool KRAKEN to try and replicate the paths noise generated by IDM would travel. This modelling, though not entirely successful, helped create some ideas for future projects. One of the key issues in modelling the source of the noise is that assuming ice generates sound as a line source or a point source is invalid. The ice sheet acts more as a 2D distributed source, generating sound that propagates into the water column and through the ice itself, essentially travelling in two directions. The other issue is that the source level of drift noise isn't quite known; its likely a combination of slipping and grinding as the ice shifts around.

Future iterations of this work could explore estimating the source level of IDM and applying that to a distributed source. Variables such as source level, the presence of the duct, and the size of the source could be altered to try and reproduce or violate the received sources in this study. Another method of validating the correlation between ANL and IDM is beamforming using data from multiple channels. This beamforming would be twofold, as each SHRU(s) is a four channel vertical array and the whole SHRU system a larger horizontal array of 5 SHRU(s). Beamforming would give insight to the directionality of sound captured, to verify is noise was coming from the area of correlation. Additionally, there is a triangular aspect to the SHRU array, as seen in \autoref{fig_location}. Time delayed responses from the triangular and linear portions of the array can be used to estimate the location of the source itself, though this method may be more suited for transient events.

Another potential idea for the future is another version of CANAPE, where some hydrophones are placed in the Duct, and some outside of it. These would provide data about sound levels both in and out of the Duct at the same time, which could be important to communications, modeling, and operations in the area. Especially as the Beaufort Duct has become a prevalent feature of the Arctic soundscape, knowing the challenges and understanding how to operate in the new environment is important for the future of Arctic research.


%%%%%%%%%%%%%%%%%%%%%%%%%%%%%%%%%%%%%%%%%%%%%%%%%%%%%%%%%%%%%%%%%%%%%%%%
\section{Importance of the Arctic}

Greater knowledge about the effects of climate change on the Arctic is critical to try to preserve this key environment of earth. These effects on this Arctic habitat fundamentally affect the life dependent on this region, including many species of fish, whales, and humans. Small changes to the environment here will surely spread through the complexities of the world ocean. The more that is known about the implications of ambient sound in an area previously covered by ice, the more the scientific community and world at large can do to try and mitigate these changes. Climate change is here to stay, and reversing the impacts of it nigh on impossible. Policies about conservation, shipping, and the exploitation of the Arctic are unfortunately one of the few ways this area can be protected from the encroaching interests of the world.