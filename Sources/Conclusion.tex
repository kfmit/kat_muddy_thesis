
%%%%%%%%%%%%%%%%%%%%%%%%%%%%%%%%%%%%%%%%%%%%%%%%%%%%%%%%%%%%%%%%%%%%%%%%
\chapter{Conclusions and Future Work}
%%%%%%%%%%%%%%%%%%%%%%%%%%%%%%%%%%%%%%%%%%%%%%%%%%%%%%%%%%%%%%%%%%%%%%%%

%%%%%%%%%%%%%%%%%%%%%%%%%%%%%%%%%%%%%%%%%%%%%%%%%%%%%%%%%%%%%%%%%%%%%%%%
\section{Statistic and Spacial Analysis conclusion}
%%%%%%%%%%%%%%%%%%%%%%%%%%%%%%%%%%%%%%%%%%%%%%%%%%%%%%%%%%%%%%%%%%%%%%%%

Exploring the probability and statistical metrics of a broad range of frequencies from 50-1900 Hz improves our understanding of the changing Arctic soundscape. For if current trends continue, soon the rest of the Arctic may resemble the badly affected Canada Basin in terms of ice loss. Thinking of the Chukchi Shelf as representative of the Beaufort Sea and the Arctic at whole allows for the consideration of the results from this study being found beyond the experimental area.  In focusing on the effects and drivers of ambient noise, generalizations about the fut

From looking at the statistic and probability of ANL for 'ice with duct', 'ice without duct', and 'no ice', 

Using different probability and statistics, certain generalizations about the soundscape of ambient noise in the Arctic can be made. Expanding the frequency of interest from 300 Hz to a significantly wider bandwidth of 1850 Hz increases the understanding of how ambient noise operates in the Arctic sea. The probability distributions of ambient noise show a 45 dB range of sound received between all three environmental conditions. As frequency increases, ANL is likely to decrease, with a minimum ambient noise around 55 dB. A clear difference between louder sound without ice and quieter sound with ice, regardless of duct, exists as well. This decibel difference changes highly with frequency, something not seen before.

From this section, it is apparent that the Beaufort duct has a strong effect on the level of noise under ice in the Arctic. The probability of the duct's presence raising the ambient noise by almost 10 dB is high when compared to a ductless environment. However, this increase in noise is limited to a certain band of frequencies of about 100 to 700 Hz. It is unlikely that the duct would cause the same ANL increase at higher frequencies. This results in similar noise levels above 100 Hz, regardless of whether the duct is present or not.

During the Arctic winter, the Arctic ice sheet dampens most of the other ambient noise normally heard on open seas. While the acoustic environment is quieter, it is by no means silent as the ice itself becomes the primary source of ambient noise.  Whether the duct is present or not, there is significant long range travel of ambient noise associated with the drift of ice. Ambient noise is captured from these areas and correlated with the drift of ice itself. This steady correlation holds from 300 to 1500 Hz as the ambient sound permeates through all these frequencies.

The size of the correlation map and the distance from SHRU5 to this spread are also linked as a larger area of ice movement creates significant ambient noise able to travel long distances. These areas of ice drift are very large, furthering their ability to send long range sound for many kilometers. Though the size of the area, distance, and the correlation coefficients of the spread between ANL and IDM differ through time, 300, 500, 1000, and 1500 Hz change together. This further demonstrating that the ambient noise encompasses many frequencies, and is highly connected to the drift of the Arctic ice sheet.

%%%%%%%%%%%%%%%%%%%%%%%%%%%%%%%%%%%%%%%%%%%%%%%%%%%%%%%%%%%%%%%%%%%%%%%%

%%%%%%%%%%%%%%%%%%%%%%%%%%%%%%%%%%%%%%%%%%%%%%%%%%%%%%%%%%%%%%%%%%%%%%%%
\section{Impacts on the Arctic}

\subsection{Biological and Habitat Impact}

The changing acoustic soundscape of the Arctic is already known to have a profound effect on the biology that inhabits the polar seas. The increase in noise that the Beaufort Duct created could affect the critical ecosystem of the Arctic by changing the habits of the life within.



\subsection{Underwater Communication}
There are some utilitarian aspects to the existence of the Beaufort Duct, if it continues to pervade the Arctic ocean over the years. The deeper duct opens the door to more long range underwater communication at lower power than before \footcite[]{freitag2015underwatercomms} . As ice-driven noise can be detected from over 1200 km away, the Beaufort duct is already known to be conducive to long range proportion. This opens the door to various scientific and anthropogenic-related interests.

The Arctic will need continued monitoring as the effects of climate change continue to reduce the ice pack. This naturally leads to a waterfall of other changes in temperature, chemical makeup, and much more. Environmental factors will need to be measured regularly in order to keep an active and correct picture of the Arctic, needed for any operations in this area. Proposals for monitoring arrays, float networks, and autonomous vehicles could benefit from the Duct's sound conducting properties by improving communications and reducing power consumption. 


%%%%%%%%%%%%%%%%%%%%%%%%%%%%%%%%%%%%%%%%%%%%%%%%%%%%%%%%%%%%%%%%%%%%%%%%
\section{Potential for Future Work}


For this project, some transmission loss estimates were made using the modal propagation tool KRAKEN to try and replicate the paths noise generated by IDM would travel. This modelling, though not entirely successful, helped create some ideas for future projects. One of the key issues in modelling the source of the noise is that assuming ice generates sound as a line source or a point source is invalid. The ice sheet acts more as a plane source, generating sound that propagates into the water column and through the ice itself, essentially travelling in two directions. The other issue is that the source level of drift noise isn't quite known; its likely a combination of slipping and grinding as the ice shifts around.

Future iterations of this work could explore estimating the source level of IDM and applying that to a plane source. Variables such as source level, the presence of the duct, and the size of the plane could be altered to try and reproduce or violate the received sources in this study. Another method of validating the correlation between ANL and IDM is beamforming using data from multiple channels. This beamforming would be twofold, as each SHRU(s) is a four channel vertical array and the whole SHRU system a larger horizontal array of 5 SHRU(s). Beamforming would give insight to the directionality of sound captured, to verify is noise was coming from the area of correlation. Additionally, there is a triangular aspect to the SHRU array, as seen in \autoref{fig_location}. Time delayed responses from the triangular and linear portions of the array can be used to estimate the location of the source itself, though this method may be more suited for transient events.

Another potential idea for the future is another version of CANAPE, where some hydrophones are placed in the duct, and some outside of it. These would provide data about sound levels both in and out of the duct at the same time, which could be important to communications, modeling, and operations in the area. Especially as the Beaufort Duct has become a prevalent feature of the Arctic soundscape, knowing the challenges and understanding how to operate in the new soundscape is important for the future of Arctic research.


%%%%%%%%%%%%%%%%%%%%%%%%%%%%%%%%%%%%%%%%%%%%%%%%%%%%%%%%%%%%%%%%%%%%%%%%
\section{Importance of the Arctic}

Knowing more about the effects of climate change on the Arctic are critical to try to preserve this key environment of earth. These effects on this Arctic habitat fundamentally affect the life dependent on this region, including many species of fish, whales, and humans. Small changes to the environment here will surely spread through the complexities of the world ocean. The more that is known about the implications of ambient sound in an area previously covered by ice, the more the scientific community and world at large can do to try and mitigate these changes. Climate change is here to stay, and reversing the impacts of it nigh on impossible. Policies about conservation, shipping, and the exploitation of the Arctic are unfortunately one of the few ways this area can be protected from the encroaching interests of the world.